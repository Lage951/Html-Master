COURSE
DEFS 
	[HOMEHTML] https://itundervisning.ase.au.dk/ITMAL_E21/Html
	[HOME]     https://itundervisning.ase.au.dk/ITMAL_E21
	[HOML]    <span style='font-family: courier new, courier;'>[HOML]</span>
	[GITMAL]  <span style='font-family: courier new, courier;'>[GITMAL]</span>
	[GITHOML] <span style='font-family: courier new, courier;'>[GITHOML]</span>
	[JPYNB]   <span style='font-family: courier new, courier;'>[JPYNB]</span>
	[OPTIONAL] (OPTIONEL)

	[KURSUSINFORMATION]  <a href='https://brightspace.au.dk/d2l/le/lessons/27524/units/244588'>kursusinformation</a>
	[KURSUSFORKORTELSER] <a href='https://brightspace.au.dk/d2l/le/lessons/27524/topics/254943' rel='noopener' target='_blank'>kursusinformation | kursusforkortelser</a>
	[KURSUSINFOGPU]      <a href='https://brightspace.au.dk/d2l/le/lessons/27524/topics/244596' rel='noopener' target='_blank'>kursusinformation | GPU Cluster</a>
	
	[BR] <br>
	
%%%%%%%%%%%%%%%%%%%%%%%%%%%%%%%%%%%%%%%%%%%%%%%%%%%%%%%%%%%%%%%%%%%%%%%%%%%%%%%
%%%%%%%%%%%%%%%%%%%%%%%%%%%%%%%%%%%%%%%%%%%%%%%%%%%%%%%%%%%%%%%%%%%%%%%%%%%%%%%

CONTENT L00

%%%%%%%%%%%%%%%%%%%%%%%%%%%%%%%%%%%%%%%%%%%%%%%%%%%%%%%%%%%%%%%%%%%%%%%%%%%%%%%
%%%%%%%%%%%%%%%%%%%%%%%%%%%%%%%%%%%%%%%%%%%%%%%%%%%%%%%%%%%%%%%%%%%%%%%%%%%%%%%

\header{Forberedelse inden kursusstart}

\sub{Formål}

\p{\i{Gruppe tilmelding:} tilmeld dig til en ITMAL gruppe (find link i Brightspace!).}

\p{\i{Installation}: de obligatoriske værktøjer til ITMAL inden kursusstart (dvs.
L01).}

\p{\i{Forberedelse til L01:} Hent GIT repositories til litteraturen [GITHOML], prøv at
kører et par Jupyter Notebooks [JPYNB], og læs mere om pythons NumPy
bibliotek.}

\p{\i{Ekstra materiale til forberedelse:} optionelle python opgaver, hvis du vil sætte
dig mere ind i sproget.}

\sub{Installation}

\ul{
	\li{Installer Anaconda på din PC:}
	\ul{
	\li{\link{https://www.anaconda.com/products/individual}}
		\li{vælg 'Download' (downloader direkte for Windows),}
		\li{eller vælg Linux eller Mac, 32 eller 64 bit (dit valg),} 
		\li{nværende nyeste Anaconda3 version er \b{2021.05}}
	}
	\ul{
		\li{ALTERNATIV 1:}
		\ul{
			\li{brug vores ASE GPU Cluster som jupyter hub server,}
			\li{se info in [KURSUSINFOGPU].}
		}
		\li{ALTERNATIV 2:}
		\ul{
			\li{Lav en konto på Google's Colaboratory,}
			\li{\link{https://colab.research.google.com}}
		}
	}
	\li{Test at du kan køre jupyter notebooks [JYPYNB] fra [GITHOML], prøv f.eks. \ipynb{index.ipynb}}
}

\sub{Forberedelse til Lektion 01}

\ul{
	\li{Læs materiale i [KURSUSINFORMATION],}
	\li{få fat i litteratur til kurset,}
	\li{clone [GITHOML] til din egen PC, se how-to under [KURSUSFORKORTELSER].}
	\li{skim denne tutorial igennem:}
	\displaystyle{\em{§ Scientific Python tutorials:} NumPy, \ipynb{tools_numpy.ipynb}, [GITHOML]
		
		[BR][BR]
		
		Læs blot, hvad du finder relevant så som 'iteration', men spring blot over
		emner, der er for komplekse eller for 'pythoniske', så som 'Stacking arrays' og
		'QR decomposition'. 
	}
}

\sub{Note vdr. kildekritik og 'informations-overload'}


\p{Vi vil i dette kurset tit kunne blive overvældet af for meget ekstern
information (informations-overload), så du skal danne dig en metode til at
kunne selektere og navigere i materialet.}

\p{Vi vil primært holde os til [HOML], [GITHOML] og Scikit-learn, med en note
om, at nettet flyder over med ekstra (til tider ubrugelig/ufiltreret)
information: en kildekritiks holdning er vigtig!} 

\sub{Ekstra materiale til forberedelse}

\p{Hvis du har brug for at opfriske dit lineær algebra matematik eller er helt ny
til python, så kan du f.eks.  læse/skimme følgende notebooks, i prioriteret
rækkefølge:}

\ol{
	\li{[OPTIONAL] python og vectors/matrices math:            [BR] \indent{\ipynb{math_linear_algebra.ipynb}        [GITHOML],}}
	\li{[OPTIONAL] python og grafisk plotting:                 [BR] \indent{\ipynb{tools_matplotlib.ipynb}           [GITHOML],}}
	\li{[OPTIONAL] ekstra, Python og dataværktøjet 'Pandas':   [BR] \indent{\ipynb{tools_pandas.ipynb}               [GITHOML],}}
	\li{[OPTIONAL] ekstra, mest for de matematik intereserede: [BR] \indent{\ipynb{math_differential_calculus.ipynb} [GITHOML].}}
}

\p{Pandas er et meget populært databehandlingsværktøj, men det
bruges/introduceres dog ikke formelt i dette kursus (du er velkommen til selv
at undersøg/bruge det).}

%%%%%%%%%%%%%%%%%%%%%%%%%%%%%%%%%%%%%%%%%%%%%%%%%%%%%%%%%%%%%%%%%%%%%%%%%%%%%%%
%%%%%%%%%%%%%%%%%%%%%%%%%%%%%%%%%%%%%%%%%%%%%%%%%%%%%%%%%%%%%%%%%%%%%%%%%%%%%%%

CONTENT L01

%%%%%%%%%%%%%%%%%%%%%%%%%%%%%%%%%%%%%%%%%%%%%%%%%%%%%%%%%%%%%%%%%%%%%%%%%%%%%%%
%%%%%%%%%%%%%%%%%%%%%%%%%%%%%%%%%%%%%%%%%%%%%%%%%%%%%%%%%%%%%%%%%%%%%%%%%%%%%%%

\header{Introduktion}

\sub{Formål}

Denne lektion har til formål at give indledende information om kurset.  Dvs. 
at vi præsentere de formelle rammer vdr.

\ul{
	\li{ITMAL gruppetilmelding,}
	\li{opgavesæt og journalafleveringer,}
	\li{eksamensform,}
	\li{Blackboard opbygning og fildeling.}
}

\p{Herefter vil vi præsentere machine learning [ML] som koncept overordnet, og
kort ridse lektionsplanen for kurset op.}

\p{Software til brug for kurset introduceres og skal installeres på jeres PC'er,
se 'L00: Forberedelse' for en installationsguide.  Vi anvender python
distributionen anaconda og i henter og installere den sidste nye version.  På
klassen vil der blive givet en kort demo af jupyter notebooks, dvs.  et at de
udviklingsværktøjer til python vi vil bruge.}

\p{Vi kigge på Scikit-learn, det primære eksterne web-sted vi vil bruge i kurset,
samt forsøge os med et par små programmer i python.}

\p{Til slut kigger vi på supervised learning og at kunne predicte
'life-satisfactory' via demo projektet i [HOML], og vi ser på pythons modul- og
klassebegreber (modules, classes), så vi kan genbruge kode i senere
lektioner..}

\sub{Indhold}

\ul{
	\li{Formelle rammer vdr. kurset.}
	\li{Eksamensform, godkendelsesfag via:}
	\ul{
		\li{et sæt obligatoriske skriftlige gruppe-journaler med afleveringsdeadlines,}
		\li{en poster-session, med aflevering af poster og mundtlig præsentation af poster,}
		\li{en mundtlig gennemgang af den sidste journal med alle medlemmer i ITMAL gruppen, samt evaluering af hver gruppemedlems
		bidrag.}
		\displaystyle{\b{\style{color: #ff3333, => Endelig godkendelse af kurset sker på en samlet vurdering af de tre punkter ovenfor.}}}
	}
	\li{Læringsmål.}
	\li{Litteratur.}
	\li{Intro til software, der bruges i ITMAL:}
	\ul{
		\li{python generelt (link til mini python intro: \link{[HOME]/L01/demo.ipynb},}
		\li{anaconda python distribution:}
		\ul{
			\li{jupyter notebooks,}
			\li{spyder developer environment.}
		}
		\li{Scikit-learn,}
		\li{opgave med python modul og klasser.}
	}
	\li{Intro til machine learning:}
	\ul{
		\li{Supervised learning (regression): 'life-satisfactory' [HOML].}
	}
}

\sub{Litteratur}

	\displaystyle{§ Preface, p. xv [HOML] (eksklusiv fra Using Code Examples...og resten af intro	kapitlet)}
	
	\displaystyle{§ 1 The machine Learning Landscape [HOML]}

	\displaystyle{§ 2 End-to-End Machine Learning Project [HOML]}

\p{Dette kapitel indeholder mange nye koncepter og en del kode.  Vi vender
senere tilbage til kapitlet senere, så læs det og prøv at danne dig et overblik
(dvs. nærlæs ikke).}

\p{Når du har installeret anaconda (se L00):}

	\displaystyle{§ Scientific Python tutorials: NumPy}
	
	\displaystyle{tools_numpy.ipynb [GITHOML]}

\p{Læs blot, hvad du finder relevant så som 'iteration', men spring blot over
emner, der er for komplekse eller for 'pythoniske', så som 'Stacking arrays' og
'QR decomposition'.}

\sub{Opgaver}

\p{Forberedelse inden lektionen}

\ul{
	\li{Meld dig ind i en ITMAL working-group [G].}
	\li{Følg installation processen givet i lektion nul ('L00: Forberedelse').}
	\li{Læs pensum.}
}

\sub{På klassen}

\ul{
	\li{Diskussion om ML (indlejret i forelæsningen).}
	\li{Opgave: intro.ipynb}
	\li{HUSK DATA til intro'en (download og udpak så "dataset" dir ligger sammen med intro.ipynb): \link{[HOME]/L01/datasets.zip}}
	\li{Opgave: modules_and_classes.ipynb}
}

\sub{Optionelle opgaver}

\p{Se 'Ekstra materiale til forberedelse' i lektion 'nul', specielt hvis du har
brug for en python og lineær algebra kick-start.}

\sub{Slides}

\displaystyle{
	\link{[HOME]/L01/lesson01.pdf}
}

END
